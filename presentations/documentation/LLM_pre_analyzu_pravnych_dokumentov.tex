\documentclass[12pt,a4paper]{article}
\usepackage[utf8]{inputenc}
\usepackage[T1]{fontenc}
\usepackage{geometry}
\usepackage{listings}
\usepackage{xcolor}
\usepackage{hyperref}
\usepackage{graphicx}
\usepackage{amsmath}
\usepackage{enumitem}

\geometry{margin=1in}

% Code listing style
\lstset{
    language=Python,
    basicstyle=\ttfamily\small,
    keywordstyle=\color{blue}\bfseries,
    stringstyle=\color{red},
    commentstyle=\color{gray}\itshape,
    numbers=left,
    numberstyle=\tiny\color{gray},
    breaklines=true,
    frame=single,
    backgroundcolor=\color{gray!10}
}


\title{Využitie LLM pre analýzu právnych dokumentov:\\Hybrid GraphRAG Documentation}
\author{Samuel Bagín}
% \supervisor{Ing. Marek Vančo, PhD.}


\begin{document}

\maketitle

\newpage
\tableofcontents
\newpage

\section{Úvod}
S nárastom využívania veľkých jazykových modelov (LLM) v rôznych oblastiach sa objavuje potreba
efektívnych metód na analýzu a spracovanie špecifických typov dokumentov, ako sú právne dokumenty.
Samotné LLM si často nevedia poradiť s komplexnými štruktúrami a vzťahmi v textoch, čo vedie k halucinácií
a obmedzeniam v ich schopnosti poskytovať presné a relevantné informácie.

\

Veľké jazykové modely uľahčujú extrakciu entít a vzťahov z textu. Alternatívne riešienie na extrakciu
entít a vzťahov by bolo možné využiť strojové učenie, avšak to by vyžadovalo trénovanie modelov na
určený jazyk, extrakciu presne stanovenej ontológie a využitie veľkého množstva dát na efektívne natrénovanie modelu.
Slovenská legislatívna doména je veľmi veľká, komplexná a členitá.

\

Ako riešenie prichádza využitie LLM na transformáciu neštruktúrovaného textu na štruktúrovné dáta.
Uloženie dát do znalostného grafu (KG) a následné využitie týchto dát na zodpovedanie otázok pomocou
metódy GraphRAG (Graph Retrieval-Augmented Generation).

\

\section{Cieľ práce}
Cieľom projektu je vytvoriť AI systém na automatickú extrakciu a
prepojenie informácií z právnych textov do znalostného grafu s
pokročilým sémantickým vyhľadávaním. Systém kombinuje grafovú
databázu s vektorovým úložiskom pre hybridné vyhľadávanie, ktoré
umožňuje používateľom klásť otázky v prirodzenom jazyku a získavať
presné odpovede na základe štruktúrovaných vzťahov aj sémantickej
podobnosti.


\newpage
\section{System Overview}

Tento projekt je postavený na pomocou jazyku Python a využíva knižnicu LangChain.

\ 

\subsection{Využívané LLM}
\begin{itemize}
    \item OpenAI
        \begin{itemize}
            \item \textbf{gpt-4o}: Model používaný na extrakciu schémy, entít a vztťahov z textu. Najbohatšia a najpresnejšia extrakcia.
            Cena (I/O pre 1M tokenov): \textbf{\$2.50 / \$10.00}
        \end{itemize}
    \item Anthropic
        \begin{itemize}
            \item \textbf{sonnet-3.5}: Model používaný na tvorenie Cypher Queries pre dotazovanie sa na grafovú databázu. Silné znalosti na tvorenie kódu a SQL, rýchly, efektívny a riešenie problémov.
            Cena (I/O pre 1M tokenov): \textbf{\$3.00 / \$15.00}
        \end{itemize}
    \item Google
        \begin{itemize}
            \item \textbf{gemini-3-flash}: Model používaný na klasifikáciu dokumentu a vytvorenie odpovede na základe získaných dát. Najlepšie pre spracovanie a zosumarizovanie veľkých kontextových okien.
            Cena (I/O pre 1M tokenov): \textbf{\$0.50 / \$3.00}
        \end{itemize}
\end{itemize}

\

\subsection{Databázy}
Pre používanie tohto projektu, používateľ si musí stiahnuť a nainštalovať aplikáciu Neo4j Desktop, a vytvoriť si lokálnu inštanciu.
\begin{itemize}
    \item Neo4j: Grafová databáza na ukladanie entít a vzťahov.
    \item Neo4jVector: Neo4j vektorová databáza na ukladanie vektorových embeddingov (vzťahy a entity) a vykonávanie vektorového vyhľadávania.
\end{itemize}


\end{document}